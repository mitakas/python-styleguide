% this is a presentation
\documentclass{beamer}

\mode<presentation>{
	% center title
	\setbeamertemplate{frametitle}{
	  %\centering
	  \smallskip
	  \insertframetitle\par
	  \smallskip
	}

	% remove controls
	\setbeamertemplate{navigation symbols}{}

	% set page number
	\setbeamertemplate{footline}[frame number]

	% remove space from caption
	\setbeamertemplate{caption label separator}{: }

	% use bullets for lists
	\setbeamertemplate{itemize item}{$\bullet$}

	% number the entries in the bibliography
	\setbeamertemplate{bibliography item}{\insertbiblabel}

	% caption without 'Figure:'
	%\setbeamertemplate{caption}{\raggedright\insertcaption\par}
	\setbeamertemplate{caption}{%
      \begin{beamercolorbox}[wd=.9\textwidth, sep=.2ex]{block body}%
        \centering%
	    \insertcaption%
      \end{beamercolorbox}%
  }

	% define some colors
	\colorlet{dark-blue}{blue!70!black}

	\setbeamercolor{frametitle}{fg=dark-blue}
	\setbeamercolor{normal text}{fg=black}
	\setbeamercolor{itemize item}{fg=dark-blue}
	\setbeamercolor{page number in head/foot}{fg=black}

	\setbeamerfont{title}{series=\bfseries,parent=structure}

	% block, light blue
	\setbeamercolor{block title}{bg=blue!20,fg=black}
	\setbeamercolor{block body}{bg=blue!5}

	% exampleblock, light green
	\setbeamercolor{block title example}{bg=green!20,fg=black}
	\setbeamercolor{block body example}{bg=green!5}

	% alertblock, light green
	\setbeamercolor{block title alerted}{bg=red!20,fg=black}
	\setbeamercolor{block body alerted}{bg=red!5}
}

% wow, wide slide
\newcommand\wider[2][3em]{%
\makebox[\linewidth][c]{%
  \begin{minipage}{\dimexpr\textwidth+#1\relax}
    \raggedright#2
  \end{minipage}%
}}

% set english as the language
\usepackage[ngerman]{babel}

% use utf-8 characters
\usepackage[utf8]{inputenc}

% use a more modern font
\usepackage{cmbright, lmodern, textcomp}

% including graphics
\usepackage{graphicx}

% colors
\usepackage{xcolor}

% drawing
\usepackage{tikz}
\usetikzlibrary{fadings}
\tikzfading[name=fade out,
  inner color=transparent!0,
  outer color=transparent!10]

% better typography
\usepackage{microtype}

% better tables
\usepackage{booktabs}

% source code listing
\usepackage{listings}

% code highlighting
\usepackage{minted}
\usemintedstyle{pastie}
\renewcommand{\theFancyVerbLine}{\ttfamily \textcolor[gray]{0.5}{\tiny \arabic{FancyVerbLine}}}

% bibliography
\usepackage{csquotes}

% add clickable links
\usepackage{hyperref}

% title, etc.
\title{Python Styleguide}
\author{Dimitar Dimitrov}
\date{\today}

\begin{document}

% -- title
\begin{frame}[plain]
  \titlepage
  \vfill
\end{frame}

\begin{frame}{Zweck und Sinn eines Styleguides}
  \begin{itemize}
    \item Zweck
    \begin{itemize}
      \item Konsistenz
      \item Programme erzeugen, die leicht
      \begin{itemize}
        \item zu \textbf{lesen},
        \item zu \textbf{verstehen},
        \item und zu \textbf{ändern} sind
      \end{itemize}
    \end{itemize}
    \item Zielgruppe
    \begin{itemize}
      \item Entwickler
    \end{itemize}
    \item Ansatz
    \begin{itemize}
      \item Richtlinien
    \end{itemize}
  \end{itemize}
\end{frame}

\begin{frame}{Styleguides und Python}
  \begin{itemize}
    \item Python Enhancement Proposals (PEPs)
      \begin{itemize}
        \item \url{https://www.python.org/dev/peps/}
      \end{itemize}
    \item PEP~8 -- Style Guide for Python Code
      \begin{itemize}
        \item \url{https://www.python.org/dev/peps/pep-0008/}
        \item Code wird viel öfter gelesen als geschrieben
        \item Es gibt Gründe, die Richtlinien zu ignorieren
      \end{itemize}
    \item PEP~20 -- The Zen of Python
      \begin{itemize}
        \item \url{https://www.python.org/dev/peps/pep-0020/}
        \item \mintinline{python}{import this}
      \end{itemize}
  \end{itemize}
\end{frame}

\begin{frame}[fragile]{pep8 vs. PEP~8}
  \begin{itemize}
    \item \texttt{pep8} ist ein Werkzeugt, das Code nach den Richtlinien in PEP~8 überprüft.
    \item \texttt{pep8} heißt inzwischen \texttt{pycodestyle}
    \item \mintinline{bash}{pycodestyle --show-source --show-pep8 FILE}
    \item \mintinline{bash}{pycodestyle --statistics -qq FOLDER}
    \item In Python-Projekte über \texttt{setup.cfg} konfiguierbar
    \begin{itemize}
      \item
\begin{minted}{yaml}
[pycodestyle]
ignore = E226,E302,E41
max-line-length = 160
\end{minted}
    \end{itemize}
    \item Mittels \mintinline{python}{# noqa} direkt in Code
  \end{itemize}
\end{frame}

\begin{frame}{Beyond PEP~8}
  \begin{itemize}
    \item Python Code Quality Authority
    \begin{itemize}
      \item \url{https://github.com/PyCQA}
      \item Werkzeuge zum Überprüfen der Qualität von Python Quellcode
    \end{itemize}
  \end{itemize}
\end{frame}

\begin{frame}{Pyflakes, Flake8, Pylint}
  \begin{itemize}
    \item Pyflakes
      \begin{itemize}
        \item \url{https://github.com/PyCQA/pyflakes}
        \item Prüft Python Quellcode nach Fehler
        \item Prüft den Stil des Codes \textbf{nicht}.
      \end{itemize}
    \item \textbf{Flake8}
      \begin{itemize}
        \item \url{https://gitlab.com/pycqa/flake8}
        \item \url{http://flake8.pycqa.org/en/latest/}
        \item Kombiniert Pyflakes und pycodestyle
      \end{itemize}
    \item Pylint
      \begin{itemize}
        \item \url{https://github.com/PyCQA/pylint}
        \item Analysiert Python Quellcode
        \item \texttt{pyreverse} generiert UML-Diagramme
      \end{itemize}
  \end{itemize}
\end{frame}

\begin{frame}{Empfehlungen}
\begin{itemize}
  \item Styleguide als formales Dokument
  \begin{itemize}
    \item Titel, Author, Datum, Version
    \item Inhaltsverzeichnis
    \item Liste der Änderungen mit Datum
    \item Seitennummer
    \item Beispiele
  \end{itemize}
  \item \texttt{pycodestyle}/\texttt{Flake8} und Git
  \begin{itemize}
    \item \mintinline{bash}{flake8 --install-hook git}
    \item \mintinline{bash}{git config --bool flake8.strict true}
  \end{itemize}
\end{itemize}
\end{frame}

\end{document}
